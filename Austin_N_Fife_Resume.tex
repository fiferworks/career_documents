%%%%%%%%%%%%%%%%%%%%%%%%%%%%%%%%%%%%%%%%%
% Short Length Professional CV
% LaTeX Template
% Version 2.0 (8/5/13)
%
% This template has been downloaded from:
% http://www.LaTeXTemplates.com
%
% Original author:
% Rishi Shah 
%
% Important note:
% This template requires the resume.cls file to be in the same directory as the
% .tex file. The resume.cls file provides the resume style used for structuring the
% document.
%
%%%%%%%%%%%%%%%%%%%%%%%%%%%%%%%%%%%%%%%%%

%---------------------------------------%
%%%%%	Packages and Contact Info   %%%%%
%---------------------------------------%
\documentclass{resume} % Use the custom resume.cls style
\usepackage[left=0.75in,top=0.6in,right=0.75in,bottom=0.6in]{geometry} % Document margins

\newcommand{\tab}[1]{\hspace{.2667\textwidth}\rlap{#1}}
\newcommand{\itab}[1]{\hspace{0em}\rlap{#1}}
\name{Austin N. Fife} % Your name
\address{\textsc{Plant Protection Entomologist}}
\begin{document}

%--------------------------------%
%%%%%	PERSONAL STATEMENT   %%%%%
%--------------------------------%
\begin{rSection}{Personal Statement}
	
I am passionate about protecting plants, and I do so leveraging my background in biological control, plant-pathogen-arthropod interactions, and chemical ecology. \end{rSection}

%----------------------------%
%%%%%%  WORK EXPERIENCE  %%%%%
%----------------------------%
\begin{rSection}{Work Experience}
	
	\textbf{Research Assistant - University of Florida} \hfill {\em \textit{2018 - Dec 2021}}\\
% - Cleaned and analyzed multidimensional data produced from analysis of volatile organic compounds via paired Gas Chromatography-Mass Spectrometry (GC-MS).\\
- Surveyed and mapped mite populations (phenology) in the field in northern Florida and southern Georgia.\\
%- Collected data at alternative sites after use of viral resources became restricted due to pest quarantine by the state agencies.\\
- Collaborated with over 10 Principal Investigators from plant pathology and entomology departments from the University of Florida, the University of Georgia, the USDA-ARS and the FDACS. Followed state and USDA permits, complied with state restrictions for movement of mites and viruses\\
 - Trained lab technicians in data entry, methodologies, and standard operating procedures.\\
 - Presented research to various audiences, including scientists, stakeholders and the public at over 11 conferences, and club meetings.\\
 - Taught class at local 4-H club, organized activities and encouraged handling of live insects during the 6th and 7th Tallahassee Science Festivals, led activities at NFREC-Quincy Arts \& Garden Festival as well as NFREC Agriculture Adventures and Ecology Field days.\\
 -  Analyzed data with statistical methods in R, including principal component analyses (PCA), uniform manifold approximation and projection (UMAP), analysis of variance (ANOVA), Chi-squared tests and Generalized Linear Mixed Modeling (GLMM). \\
 %- Authored and published 2 journal articles (one in revisions), a technical report, as well as multiple lab protocols and standard operating procedures with RMarkdown.\\
 %- Presented chapters on support vector machines (SVM) and PCA for university book club about Data Science, AI, Machine Learning, and Deep Learning in R.

\textbf{Research Assistant - University of Idaho} \hfill {\em \textit{2015 - 2018}}\\
- Studied host plant selection of the potato psyllid \textit{Bactericera cockerelli}, vector of Zebra Chip Disease in potato.\\
- Maintained four insect colonies, grew 100+ plants in the greenhouse, including 10 varieties of potato, tomatoes, eggplant, weeds and various species of native plants from seed\\
- Used statistical methods in R to interpret data, including t-tests, ANOVA, and GLMMs.\\
- Developed protocols and standard operating procedures (SOPs), to record insect behaviors and fecundity on living plants with a limited budget.\\
- Assisted in pest monitoring by processing hundreds of sticky traps weekly.\\
- Wrote and published thesis with Latex.\\
- Manually sorted and weighed 70+ lb sacks of potatoes with a small team.\\

%\newpage
\textbf{Research \& Teaching Assistant - Brigham Young University - Idaho} \hfill {\em \textit{2013 - 2015}}\\
- Drove an Off-road vehicle to navigate sand dunes and record coordinates of \textit{Cicindela arenicola}, the St. Anthony Dunes Tiger Beetle to study habitat characteristics, range, and dispersal.\\
- Independently obtained permit from National Park Service and the Bureau of Land Management Craters of the Moon National Monument to collect data on a threatened species of beetle, \textit{Glacicavicola bathysciodes}\\
- Prepared daily labs and mentored students\\
- Tutored Spanish language conversation labs\\
	\hfill
\end{rSection}


%--------------------%
%%%%%%  SKILLS   %%%%%
%--------------------%
\begin{rSection}{Skills}
\textbf{Problem Solving} • \textbf{Collaboration} • \textbf{Research} • \textbf{Writing} • \textbf{Project Management}\\
Languages --- \textbf{Spanish, English} \hfill Programming --- \textbf{R, Latex, RMarkdown}\\   
Software --- \textbf{RStudio, Anaconda, Git, Microsoft Office, Adobe Photoshop}\\
Hardware --- \textbf{Raspberry Pi, GC-MS, Phase-Contrast Microscopy, Digital Cameras}\\
Operating Systems --- \textbf{Windows 10, Linux}
\end{rSection}


%-----------------------%
%%%%%	EDUCATION   %%%%%
%-----------------------%
\begin{rSection}{Education}
	\textsc{PhD: Entomology - University of Florida} \hfill {\em Anticipated: Dec 2021}\\
	\textit{\textbf{Dissertation:}} Mite-virus-plant complexes of importance for Florida agriculture: early detection, chemical ecology and biocontrol of \textit{Phyllocoptes fructiphilus} and \textit{Brevipalpus californicus} \\
	\textit{\textbf{Courses:}} Plant-Pathogen-Insect Interactions, Agricultural Acarology, Insect Classification,\\
	Introduction to Acarology, Insect Chemical Ecology, Epidemiology \& Data Science,\\
	Spatial Ecology of Insects, Introduction to Applied Statistics, Data Storytelling,\\
	Ecology of Vector-Borne Disease, Vector Biology Models, Insect Microbiology\\ \hfill

	\textsc{MSc: Entomology - University of Idaho} \hfill {\em 14 Dec 2018}\\ \textit{\textbf{Thesis:}} Investigating behavior of the potato psyllid \textit{Bactericera cockerelli} (Šulc)\\
	(Hemiptera: Triozidae) on three potato genotypes with putative resistance to\\ “\textit{Candidatus} Liberibacter solanacearum”\\
	\textit{\textbf{Courses:}} Insect-Plant Interactions, Host Plant Resistance, Plant Pathology, Advanced Insect Ecology, Advanced Forest Entomology, Insect Physiology, Potato Science\\ \hfill
	
	\textsc{BS: Zoology - Brigham Young University - Idaho} \hfill {\em 10 Apr 2015}\\ \textit{\textbf{Courses:}} Insect Systematics, General Entomology, Biochemistry \& Molecular Biology,\\
	An Evolutionary Survey of Plants, General Botany, Biostatistics, Understanding DNA,\\
	Evolutionary Science, Genetics and Molecular Biology, Invertebrate/Vertebrate Zoology,\\
	General Chemistry I, General Chemistry II,  Ecology I, Potato Science\\
	Readings in Hispanic Literature - Advanced Speaker\\ \hfill
\end{rSection}


%-------------------------
%%%%%%  PUBLICATIONS %%%%%
%-------------------------
\begin{rSection}{Publications}
\textit{Journal of Integrated Pest Management} \hfill {\em \textit{Accepted with revisions}}\\
`First report of the \textit{Brevipalpus}-transmitted (Trombidiformes: Tenuipalpidae)\\
\textit{Orchid fleck dichorhavirus} infecting three ornamental in Florida'\\
\textbf{Austin N. Fife}, Daniel Carrillo, Gary Knox, Fanny Iriarte, Kishore Dey, Avijit Roy,\\
 Ronald Ochoa, Gary Bauchan, Mathews Paret, and Xavier Martini \hfill
 
\textit{Florida Entomologist} \hfill {\em \textit{Sep 2020}}\\
`First Report of \textit{Phyllocoptes fructiphilus} Kefier (Eriophyidae), the vector of\\
the rose rosette virus, in Florida, USA'\\
\textbf{Austin N. Fife}, Samuel Bolton, Jessica L. Griesheimer, Mathews Paret, and Xavier Martini \hfill

\textit{Journal of Insect Science} \hfill {\em \textit{Mar 2020}}\\
`Potato psyllid \textit{Bactericera cockerelli} (Šulc) (Hemiptera: Triozidae) behavior on\\
 three potato genotypes with putative resistance to ``\textit{Candidatus} Liberibacter solanacearum"'\\
\textbf{Austin N. Fife}, Arash Rashed, Regina Cruzado Gutierrez,\\ Richard Novy, and Erik J. Wenninger

\end{rSection}
\end{document}
