%%%%%%%%%%%%%%%%%%%%%%%%%%%%%%%%%%%%%%%%%
% Short Length Professional CV
% LaTeX Template
% Version 2.0 (8/5/13)
%
% This template has been downloaded from:
% http://www.LaTeXTemplates.com
%
% Original author:
% Rishi Shah 
%
% Important note:
% This template requires the resume.cls file to be in the same directory as the
% .tex file. The resume.cls file provides the resume style used for structuring the
% document.
%
%%%%%%%%%%%%%%%%%%%%%%%%%%%%%%%%%%%%%%%%%

%----------------------------------------
%%%%%	Packages and Contact Info   %%%%%
%----------------------------------------
\documentclass{resume} % Use the custom resume.cls style

\usepackage[left=0.75in,top=0.5in,right=0.75in,bottom=0.5in]{geometry} % Document margins

\usepackage[document]{ragged2e}

%\newcommand{\tab}[1]{\hspace{.2667\textwidth}\rlap{#1}}
%\newcommand{\itab}[1]{\hspace{0em}\rlap{#1}}
\name{Austin N. Fife} % Your name
%\address{afife@ufl.edu \\ 3216 Nekoma Ln Apt. B, Tallahassee, FL 32304, USA \\ (208) 874-2283} % Your address
\begin{document}

%-----------------------------------
%%%%%	Statement of Purpose   %%%%%
%-----------------------------------
\begin{rSection}{Research Statement}

%Biocontrol
%Eriophyid mites - well adapted to plants
%Spatial Ecology
My main areas of expertise are in the ecology of arthropod-plant-pathogen interactions,\\ chemical ecology, arthropod identification, data analysis, statistical methods, and\\ programming in R.

\end{rSection}

%------------------------
%%%%%	Education   %%%%%
%------------------------
\begin{rSection}{Education}
	
	\textsc{PhD: Entomology - University of Florida} \hfill {\em Anticipated date: Dec 2021}\\
	\textit{Proposal:} Distribution and biocontrol of \textit{Phyllocoptes fructiphilus} and \textit{Brevipalpus californicus}\\ (Acari: Tetranychoidea): mite vectors of plant viruses of ornamental plants in Florida \\
	\textit{Courses:} Epidemiology \& Data Science, Spatial Ecology of Insects, Introduction to Applied Statistics, Data Storytelling, Insect Chemical Ecology, Ecology of Vector-Borne Disease,\\
	Vector Biology Models, Plant-Pathogen-Insect Interactions, Agricultural Acarology,\\
	Introduction to Acarology, Insect Classification, Insect Microbiology\\ \hfill
	
	\textsc{MSc: Entomology - University of Idaho} \hfill {\em 2018}\\
	\textit{Thesis:} Investigating behavior of the potato psyllid \textit{Bactericera cockerelli} (Šulc) (Hemiptera: Triozidae) on three potato genotypes with putative resistance to “\textit{Candidatus} Liberibacter solanacearum” - \textit{Journal of Insect Science, Volume 20, Issue 2, March 2020, 15}\\
	\textit{Courses:} Advanced Insect Ecology, Advanced Forest Entomology,
	Insect-Plant Interactions, Host Plant Resistance, Plant Pathology, Insect Physiology, Potato Science\\ \hfill
	
	\textsc{BS: Zoology - Brigham Young University - Idaho} \hfill {\em 2015}\\ \textit{Courses:} Biostatistics, Ecology I, Invertebrate/Vertebrate Zoology, Ornithology, Mammalogy, An Evolutionary Survey of Plants, General Botany, Insect Systematics, General Entomology, Evolutionary Science, Understanding DNA, Genetics and Molecular Biology,\\ Biochemistry \& Molecular Biology, General Chemistry I, General Chemistry II,\\ Potato Science, Readings in Hispanic Literature - Advanced Speaker\\ \hfill
	
\end{rSection}

\newpage

%----------------------
%%%%%%  RESEARCH  %%%%%
%----------------------
\begin{rSection}{Research Experience}

\textsc{Research Assistant - University of Florida} \hfill {\em \textit{2018 - Dec 2021}}\\
Studied changes in chemistry (volatile organic compounds), for virus-infected plants via\\ headspace volatile collection with Solid Phase Microextraction (SPME) with paired\\
Gas Chromatography/Mass Spectrometry. Interpreted results with statistical methods in R, including principal component analyses (PCA), uniform manifold approximation\\
and projection (UMAP), analysis of variance (ANOVA), Chi-squared tests and Generalized Linear Modeling. Discovered two previously unreported mite-plant-virus complexes in Florida, one article has been published and the other paper is accepted with minor revisions\\ (see Publications). Monitored and mapped mite populations (phenology) in the field.\\
Collected data at alternative sites after use of viral resources became restricted due to pest quarantine by the FDACS. Collaborated with over 10 PIs from plant pathology and\\
entomology departments from the University of Florida, the University of Georgia, the USDA-ARS and the FDACS. Followed state and USDA permits, complied with state restrictions for movement of mites and viruses and developed unique solution to collect samples by developing field method for extracting volatiles on site via SPME in the field. Taught and worked with seasonal personnel to conduct experiments and maintain projects. Developed 15+ standard operating procedures (SOPs), protocols and improved methods for surveying, processing\\
samples, preparing chemistry, and mounting mites. Grew over 100 roses, beans, tobacco,\\ orchids, liriopogons, and etc. in a greenhouse. Maintained temporary mite colonies.\\
Conducted predatory mite olfactometer behavioral assays and followed
protocols and methods for qPCR and RPA.  Suppressed herbivorous mites (\textit{Phyllocoptes fructiphilus}) in three field\\
<<<<<<< HEAD
trials by combining biocontrol methods with predatory mites and systemic acquired\\ resistance. Established colonies of herbivorous mites. Presented at an extension workshops and local rose club. Involved in course on impact network analysis and a machine learning book club at the University of Florida related to Data Science, AI, Machine Learning, and Deep Learning, where members present and discuss deep learning applications in R. \\
=======
trials by combining biocontrol methods with predatory mites and systemic acquired\\ resistance. Established colonies of herbivorous mites. Presented at an extension workshops and local rose club. Involved in course on impact network analysis and a machine learning bookclub at the University of Florida related to Data Science, AI, Machine Learning, and Deep Learning, where members present and discuss deep learning applications in R. \\
>>>>>>> 81fd932ebfcc17498db58021da2bb7ac7e8fe419
Have current pesticide applicator's permit for ornamental plants and turf in Florida.
\hfill

\textsc{Research Assistant - University of Idaho} \hfill {\em \textit{2015 - 2018}}\\
Studied host plant selection of the potato psyllid \textit{Bactericera cockerelli}, vector of Zebra Chip Disease in potato. Maintained four insect colonies, grew hundreds of plants in the greenhouse, including 10 varieties of potato, tomatoes, eggplant, weeds and various species of native plants from seed to test psyllid feeding behaviors. Used statistical methods in R to interpret results,\\
including t-tests, ANOVA, and generalized linear mixed models. Wrote and published\\ thesis with Latex. Developed protocols, standard operating procedures (SOPs), and\\ methods to record insect behaviors and fecundity on living plants with a limited budget.\\
Assisted in pest monitoring by processing hundreds of sticky traps weekly. Manually sorted and weighed potatoes with a small team.\\ \hfill

\textsc{Research \& Teaching Assistant - Brigham Young University - Idaho} \hfill {\em \textit{2013 - 2015}}\\
Drove an Off-road vehicle and navigated sand dunes with GPS to study habitat\\
characteristics, range, and dispersion of \textit{Cicindela arenicola}, the St. Anthony Dunes\\ Tiger Beetle. Prepared lab activities, graded papers, assisted students and curated the insect collection. Independently applied for and obtained permit from National Park Service and the Bureau of Land Management Craters of the Moon National Monument to collect data on a threatened species, \textit{Glacicavicola bathysciodes}, the Western Ice Cave Beetle.\\
Spelunked in the Civil Defense caves to observe \textit{G. bathysciodes}\\
Tutored Spanish class with conversation practice.\\ \hfill


%-------------------------
%%%%%%  PUBLICATIONS %%%%%
%-------------------------
\subsection*{Publications}
\textit{Journal of Integrated Pest Management} \hfill {\em \textit{Accepted with revisions}}\\
`First report of the \textit{Brevipalpus}-transmitted (Trombidiformes: Tenuipalpidae)\\
\textit{Orchid fleck dichorhavirus} infecting three ornamental in Florida'\\
\textbf{Austin N. Fife}, Daniel Carrillo, Gary Knox, Fanny Iriarte, Kishore Dey, Avijit Roy,\\
 Ronald Ochoa, Gary Bauchan, Mathews Paret, and Xavier Martini\\ \hfill
 
\textit{Florida Entomologist} \hfill {\em \textit{Sep 2020}}\\
`First Report of \textit{Phyllocoptes fructiphilus} Kefier (Eriophyidae), the vector of\\
the rose rosette virus, in Florida, USA'\\
\textbf{Austin N. Fife}, Samuel Bolton, Jessica L. Griesheimer, Mathews Paret, and Xavier Martini\\ \hfill

\textit{Journal of Insect Science} \hfill {\em \textit{Mar 2020}}\\
`Potato psyllid \textit{Bactericera cockerelli} (Šulc) (Hemiptera: Triozidae) behavior on\\
 three potato genotypes with putative resistance to ``\textit{Candidatus} Liberibacter solanacearum"'\\
\textbf{Austin N. Fife}, Arash Rashed, Regina Cruzado Gutierrez,\\ Richard Novy, and Erik J. Wenninger\\

%----------------------------
%%%%%%  PRESENTATIONS  %%%%%%
%----------------------------

\subsection*{Oral Presentations}
Plant Health 2021 - The American Phytopathological Society (Online) \hfill {\em \textit{Aug 2021}}\\
`New encounters with old problems: Orchid fleck dichorhavirus infecting three new ornamental hosts in Florida'\\
\textbf{Austin N. Fife}, Xavier Martini, Mathews Paret, Gary Knox, Kishore Dey,\\
Gary R. Bauchan, and Ronald Ochoa\\ \hfill

2021 Virtual Southeastern Branch Meeting (Online) \hfill {\em \textit{Mar 2021}}\\
`\textit{Orchid fleck dichorhavirus}: A new \textit{Brevipalpus} transmitted virus fresh from Florida'\\
\textbf{Austin N. Fife}, Xavier Martini, Mathews Paret, Gary Knox, Kishore Dey,\\
Gary R. Bauchan, and Ronald Ochoa\\ \hfill

Annual Meeting of the of the Entomological Society of America (Online) \hfill {\em \textit{Nov 2020}}\\
`Exploring new areas of integrated pest management for mites on roses:\\
predatory mites and systemic acquired resistance'\\
\textbf{Austin N. Fife}, Gary Knox, Xavier Martini, and Mathews Paret\\ \hfill

Annual Meeting of the of the Entomological Society of America, St. Louis, MO \hfill {\em \textit{Nov 2019}}\\
`\textit{Amblyseius swirskii} attraction to volatiles produced from roses\\
infected with Rose Rosette Virus'\\
\textbf{Austin N. Fife}, Gary Knox, Xavier Martini, and Mathews Paret\\ \hfill

\newpage

Annual Meeting of the Florida Entomological Society, St. Augustine, FL \hfill {\em \textit{Jul 2018}}\\
`Settling behaviors of the potato psyllid, \\ \textit{Bactericera cockerelli} (Hemiptera: Triozidae), on different germplasms'\\
\textbf{Austin N. Fife}, Arash Rashed, Regina Cruzado Gutierrez,\\
Richard Novy, and Erik J. Wenninger\\ \hfill

Annual Meeting of the Pacific Branch of \hfill {\em \textit{Apr 2017}}\\
the Entomological Society of America, Portland, OR\\
`Observing the settling behavior of the potato/tomato psyllid,\\
\textit{Bactericera cockerelli} (Šulc) on different potato germplasms'\\
\textbf{Austin N. Fife}, Arash Rashed, Regina Cruzado Gutierrez,\\
Richard Novy, and Erik J. Wenninger\\

%----------------------
%%%%%%  POSTERS  %%%%%%
%----------------------

\subsection*{Poster Presentations}

ECOIPM - Ornamental Workshop on Diseases and Insects, Hendersonville, NC \hfill {\em \textit{Oct 2018}}\\
`Preliminary volatile analysis from RRV - infected roses'\\
\textbf{Austin N. Fife}, Xavier Martini, and Mathews Paret\\ \hfill

XXV International Congress of Entomology, Orlando, FL  \hfill {\em \textit{Sep 2016}}\\
`Settling behavior of the potato psyllid, \textit{Bactericera cockerelli} (Šulc), (Hemiptera: Triozidae) on potato germplasm with putative resistance to \textit{Candidatus} Liberibacter solanacearum (Lso)'\\
\textbf{Austin N. Fife}, Arash Rashed, Regina Cruzado Gutierrez, Richard Novy, and\\ Erik J. Wenninger \hfill

Entomological Society of America - 62nd Annual Meeting, Portland, OR \hfill {\em \textit{Nov 2014}}\\
`Population Survey of St. Anthony Dunes tiger beetle, \textit{Cicindela arenicola} (Coleoptera: Cicindelidae)'\\
\textbf{Austin N. Fife}, Ismael E. Ramirez, and John Zenger\\ \hfill

\end{rSection}

\newpage

%-----------------------
%%%%%%  EXTENSION  %%%%%
%-----------------------

\begin{rSection}{Extension Activities}
	My goal for agricultural extension activities is to address the concerns of\\
	growers, shareholders, and the public. I believe that empathy and relationships are\\
	crucial	for adoption of best management practices. By drawing upon their collective\\
	knowledge, we can collaborate to enhance agriculture for the benefit of all. 
	
	I have published an extension article, presented at local clubs twice, and directed	activities which augmented participants' knowledge of beneficial insects, pests, and their control:\\
	Presented collections of insect pests at the NFREC-Quincy Arts \& Garden Festival.\\
	Taught a K-12 and adults audience about arthropod body plans.\\
	Demonstrated insect pinning methods	to preteens and discussed insect collecting at local 4-H meetings of Insect Club. Demonstrated a `build-a-bug' activity and encouraged handling of live insects during the 6th and 7th Tallahassee Science Festivals, compared generalists	to\\
	specialist predators as learning activity for first grade students for the North Florida Research and Education Center - Agriculture Adventures and Ecology Field days.
	
	\subsection*{Extension Publications}
	\textit{UF IFAS Extension Publication - Gardening in the Panhandle} \hfill {\em \textit{Aug 2020}}\\
	`Help Us Keep A Watch Out for Rose Rosette Disease -\\ 
	\textit{Phyllocoptes fructiphilus}: a new threat for Florida roses'\\
	\textbf{Austin N. Fife}, Gary Knox, Mathews Paret, and Xavier Martini\\ \hfill

	\subsection*{Oral Presentations}
	New Threats from Rose Rosette Virus and the Eriophyid Mite Vector, Tallahassee, FL \hfill {\em \textit{2020}}\\
	`Discovery of the RRD eriophyid mite vector in Florida and mite history and biology'\\
	\textbf{Austin N. Fife}, Gary Knox, Xavier Martini, and Mathews Paret\\ \hfill
	
	Presentation for the Tallahassee Rose Society, Tallahassee, FL \hfill {\em \textit{Jan 2020}}\\
	`Rose Rosette Disease'\\
	\textbf{Austin N. Fife}, Gary Knox, Xavier Martini, and Mathews Paret\\ \hfill
	
\end{rSection}


%---------------------
%%%%%%  Grants, Awards \& Skills   %%%%%
%---------------------
\begin{rSection}{Grants, Awards \& Skills}

Florida Nursery, Growers and Landscape Association Endowed Research Fund \hfill {\em \textit{\$4,956}}\\
`Survey of the invasive mite \textit{Phyllocoptes fructiphilus}, Rose Rosette Virus (RRV),\\
 and predatory mites in Northern Florida'\\
\textbf{Austin N. Fife} and Xavier Martini

APS Foundation Virology Travel Award - Plant Health 2021 (Online) \hfill {\em \textit{2021}}\\
Abstract Award - Acarological Society of America \hfill {\em \textit{2019}}\\
Master's Student Paper Competition - 3rd Place - Florida Entomological Society \hfill {\em \textit{2018}}\\
Master's Student Paper Competition - 2nd Place - Pacific Branch ESA \hfill {\em \textit{2017}}\\
Student Travel Grant - Pacific Branch ESA \hfill {\em \textit{2017}}\\
Spanish - oral and written \hfill {\em \textit{Fluent}}\\
Programming Languages - R, LaTex \hfill {\em \textit{Proficient}}\\

\end{rSection}

\end{document}
