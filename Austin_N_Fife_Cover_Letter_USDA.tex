%%%%%%%%%%%%%%%%%%%%%%%%%%%%%%%%%%%%%%%%%
% Plain Cover Letter
% LaTeX Template
% Version 1.0 (28/5/13)
%
% This template has been downloaded from:
% http://www.LaTeXTemplates.com
%
% Original author:
% Rensselaer Polytechnic Institute 
% http://www.rpi.edu/dept/arc/training/latex/resumes/
%
% License:
% CC BY-NC-SA 3.0 (http://creativecommons.org/licenses/by-nc-sa/3.0/)
%
%%%%%%%%%%%%%%%%%%%%%%%%%%%%%%%%%%%%%%%%%

%----------------------------------------------------------------------------------------
%	PACKAGES AND OTHER DOCUMENT CONFIGURATIONS
%----------------------------------------------------------------------------------------

\documentclass[11pt]{letter} % Default font size of the document, change to 10pt to fit more text

%\usepackage{newcent} % Default font is the New Century Schoolbook PostScript font 
\usepackage{helvet} % Uncomment this (while commenting the above line) to use the Helvetica font

\usepackage{microtype}

% Margins
\topmargin=-1.5in % Moves the top of the document 1 inch above the default
\textheight=8.5in % Total height of the text on the page before text goes on to the next page, this can be increased in a longer letter
\oddsidemargin=-10pt % Position of the left margin, can be negative or positive if you want more or less room
\textwidth=6.5in % Total width of the text, increase this if the left margin was decreased and vice-versa

\let\raggedleft\raggedright % Pushes the date (at the top) to the left, comment this line to have the date on the right

\begin{document}
	\begin{letter}
		%----------------------------------------------------------------------------------------
		%	LETTER CONTENT SECTION
		%----------------------------------------------------------------------------------------
		\\
		Dear members of the search committee,\\
		
		My name is Austin N Fife, and I believe that my experience with acarology, entomology, plant-arthropod-pathogen-interactions, and integrated pest management/biological control can help your agency to control invasive plants.
		
		Over the past six years, I have designed and orchestrated multiple field trials, as well as lab experiments to protect row crops and ornamentals from psyllids, mites, and the pathogens they vector. The majority of my research focused on integrating different management methods to control vectors of plant pathogens with natural enemies, plant defenses, and sanitation measures. My PhD research was based on the early detection, chemical ecology and biocontrol of \textit{Phllocoptes fructiphilus} and \textit{Brevipalpus californicus}, (partially funded by USDA-AFRI-CPPM 2017-70006-27268 and the USDA-NIFA, Hatch project FLA-NFC-005607). These two mite species are vectors of economically important plant pathogens, and I was the first to detect and report on these pathosystems in northern Florida. I also worked with chemical ecology methods to determine predator attraction to plant volatiles, hoping to improve predatory efficiency or develop chemical lures. I also studied the fecundity and behavior of \textit{Bactericera cockerelli} during my master's degree to evaluate host plant resistance (funded by USDA-NIFA 2014-67014-22408). I have created/maintained various insect and mite colonies, as well as growing hundreds of plants from an eclectic variety of species. I have an enthusiasm for statistical programming in R, which I have enjoyed over the past six years. I have a current pesticide applicator license, and valid drivers license for Florida.
		
		I take a multidisciplinary approach to problem-solving, accordingly, I have collaborated as frequently as possible to learn from others. I have benefited from the combined expertise of over 15 principal investigators from regional universities, as well as the Florida Department of Agriculture and Consumer Services, and USDA-APHIS-PPQ. Furthermore, working at Research and Education Centers over the last six years has helped me to understand the needs of different organizations: I have spoken at 4-H and local rosarian clubs, presented to stakeholders, published technical reports, organized and led educational activities, set up information booths and participated in various other STEM events to share our research with local communities. I also speak, read, and write Spanish fluently, due to living independently in Mexico for a few years, and I appreciate interacting with diversified communities. I also understand the importance of maintaining a positive lab culture while working with and supervising technical staff: I was a visiting student researcher in Georgia for a year, so I would be gone for days working on independent projects while lab technicians back in Florida followed my protocols and standard operating procedures.
		
		I have written extension articles, journal articles, proposals, and received a small grant for our research. I consistently speak at annual professional meetings by various scientific organizations, in person and online, including oral and poster sessions. I am first author on three publications in peer-reviewed journals, and I expect to finish 2-3 more publications in the near future.
		
		In conclusion, I am confident that my experience with biological control, arthropod-plant-pathosystems, and  makes me a strong candidate for your agency.
		
		Thank you for your time and consideration,
		
		\quad -- Austin Nathaniel Fife
		
		\thispagestyle{empty}
		%----------------------------------------------------------------------------------------
		
	\end{letter}
	
\end{document}\textbf{}