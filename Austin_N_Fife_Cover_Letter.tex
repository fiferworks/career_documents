%%%%%%%%%%%%%%%%%%%%%%%%%%%%%%%%%%%%%%%%%
% Plain Cover Letter
% LaTeX Template
% Version 1.0 (28/5/13)
%
% This template has been downloaded from:
% http://www.LaTeXTemplates.com
%
% Original author:
% Rensselaer Polytechnic Institute 
% http://www.rpi.edu/dept/arc/training/latex/resumes/
%
% License:
% CC BY-NC-SA 3.0 (http://creativecommons.org/licenses/by-nc-sa/3.0/)
%
%%%%%%%%%%%%%%%%%%%%%%%%%%%%%%%%%%%%%%%%%

%----------------------------------------------------------------------------------------
%	PACKAGES AND OTHER DOCUMENT CONFIGURATIONS
%----------------------------------------------------------------------------------------

\documentclass[11pt]{letter} % Default font size of the document, change to 10pt to fit more text

%\usepackage{newcent} % Default font is the New Century Schoolbook PostScript font 
\usepackage{helvet} % Uncomment this (while commenting the above line) to use the Helvetica font

\usepackage{microtype}

% Margins
\topmargin=-1.5in % Moves the top of the document 1 inch above the default
\textheight=8.5in % Total height of the text on the page before text goes on to the next page, this can be increased in a longer letter
\oddsidemargin=-10pt % Position of the left margin, can be negative or positive if you want more or less room
\textwidth=6.5in % Total width of the text, increase this if the left margin was decreased and vice-versa

\let\raggedleft\raggedright % Pushes the date (at the top) to the left, comment this line to have the date on the right

\begin{document}
\begin{letter}
%----------------------------------------------------------------------------------------
%	LETTER CONTENT SECTION
%----------------------------------------------------------------------------------------
\\
Dear members of the search committee,\\

My name is Austin N Fife. I believe that my experience with chemical ecology, plant-herbivore interactions, and statistical analysis in R makes me a suitable candidate for your program.

The majority of my research has focused on studying how mite vectors of plant pathogens interact with their host plants, natural enemies, and plant defenses (induced systemic acquired resistance, SAR). My PhD research was based on the early detection, chemical ecology and biocontrol of \textit{Phllocoptes fructiphilus} and \textit{Brevipalpus californicus}. I have experience using paired Gas Chromatography - Mass Spectrometry for analyzing plant compounds: I used volatile collection traps and solid phase microextraction methods to extract volatile organic compounds released from the headspace of roses, and evaluated their attractiveness to predatory mites. I also studied the effects of inducing SAR on headspace volatiles and mite populations, providing preliminary data to improve predatory efficiency and/or develop chemical lures. During the past six years, I have conducted a variety of field, lab and greenhouse trials: I have grown hundreds of plants from an eclectic variety of plant groups, including many species of wild and cultivated Solanaceae. Although I have not done non-targeted metabolomics analysis, I have experience with various machine learning algorithms and exposure to network analysis methods in R, so I am confident that I could quickly learn the skills needed to perform such an analysis.

I take a multidisciplinary approach to problem-solving, accordingly, I have collaborated as frequently as possible to learn from others. I have benefited from the combined expertise of over 15 principal investigators from regional universities, as well as state and governmental agencies. I have both lab and fieldwork experience. I also speak, read, and write Spanish fluently, due to living independently in Mexico for a few years, and I enjoy interacting with diverse communities. 

I have written extension articles, journal articles, proposals, and received a few small grants as described in my CV. I speak consistently at annual professional meetings by various scientific organizations, in person and online, including oral and poster sessions. I am first author on three publications in peer-reviewed journals, and I expect to finish 2-3 more publications in the near future.

In conclusion, I am confident that my experience with chemical ecology, plant defenses, and arthropod-plant-pathosystems makes me a viable candidate for your lab.

Thank you for your time and consideration,

\quad -- Austin N Fife

\thispagestyle{empty}
%----------------------------------------------------------------------------------------

\end{letter}

\end{document}